\documentclass[UTF8, b5paper]{book}

\usepackage{ctex}

\title{收获与播种\\对一个数学家过去的回顾和见证}
\author{[法]亚历山大·格罗滕迪克\ 著\\cpz\ 译}
\date{2022.11.30}

\begin{document}
\maketitle
\tableofcontents

\chapter*{}*{主题介绍, 或是说四个乐章的前奏.}
\addcontentsline{toc}{chapter}{主题介绍, 或是说四个乐章的前奏.}
\section{写在前面的话}
1986年1月30日 \par
现在只剩下前言部分没写了. 写完它, 我就可以把《收获与播种》付梓了.我发誓, 我有世界上最好的意愿来写一些有用的东西.%fasse l'affaire.
这次, 我要写点\textbf{有道理}的东西. %Quelque chose de raisonnable, cette fois.
前言的三四页, 对于呈现这个上千页的巨大基石(pavé)来说确实不多, 但是够精深了.
这次, 我要写点能吊住那些不耐烦的读者胃口的东西, 使他们有可能在这上千页篇幅中发现点有意思的东西(没准和他们自己有关, 谁知道呢?).我通常不会主动去吊足读者胃口, 这不是我的风格. 但这次不一样, 我打算破例一次! 那些蠢到敢接下来(出版这个上千页的、可能压根不适合出版的“怪物)这样的活的出版商, 他们会为此付出代价的. \par
然而, 尽管我尽了全力, 这一切却并没有发生. 我预计我能用一个下午快速地写完它, 但实际上并没有. 到了明天, 我就会在它上面耗费三个星期了, 我的稿纸都要堆成小山了. 接下来的, 确切地说, 已经不能够称之为一个体面的"前言"了. 没错——我又写砸了. 你在我的年纪可不能再这样做了. 我没有卖书或者找人帮我卖书的打算, 即使这样能给(自己、好友)带来好处.
接下来的, 便是对于我数学家生涯的一种"漫记", 还是给那些"门外汉"们的"漫记"——我口中的“门外汉”,就是指的那些几乎对数学一无所知的人。同时,这也是给我的,我可从来都没有空来闲逛,来写这种“漫记”。我渐渐能把我之前没能说出口的事情一个接一个地讲出来了,包括(在我看来)我劳作过程中的一些“精粹”。这些事情和具体的技术没什么关系。我传递这些东西的事业(这有点疯狂!)是否成功,可以让你说了算。如果你能感受到一些什么,我会很知足、很快乐的。这可能是我那些渊博的同事们一辈子都感受不到的东西。他们过于渊博、过于谨慎了,而这恰恰会使他们与最简单最精粹的东西失之交臂。\par
在这本“工作漫记”之中,我也会简单谈谈我个人的生活。除此,我还会简单说说《收获与播种》简单说了些什么,我也会在“漫记”之后的“信件”部分更加仔细地谈谈这件事。“信件”是写给我的学生们和我在数学界的“老朋友”们的,但也没什么技术性可言。任何想了解对于我创作《收获与播种》的来龙去脉的读者,都可以从我的第一视角的叙述中毫不费力地读下去。甚至,相比于“漫记”,它能够给你关于数学的森罗万象的一点点特定的感触,也能给你一点关于我的表达风格的感触(就像我在“漫记”里做的那样),这可能稍微显得有些特别。而这种风格所体现出的一种精神——也不是普罗大众都能欣赏的精神。\par
在“漫记”中,我会谈谈\textbf{数学工作}。在《收获与播种》全篇中,我也会提一点。这是我熟知的、也是第一手的工作。而且,自然地,我的大部分想法对于所有创造性的、发现性的工作也都适用。至少,这对于所有“智性”工作都适用——所谓智性工作,是指那些主要在头脑里做完然后写出来的工作。这种工作的一大特征就是我们对于所探知事物的\textbf{认知}的涌现。但是,考虑一个相反的例子:爱的激情亦是探索的动力。它可以展现给我们“爱欲使然的”知识,这种知识是历久弥新的:它开花散叶、日渐深宏。这两种冲动——其一,让我们这样说,是推进数学家工作的冲动;其二,则是那种坠入爱河之人具有的冲动——它们之间的关联要远比我们所怀疑的或者所设想的那样更近。我希望《收获与播种》能够帮助你在你的工作或者是日常生活中体会到它。\par
在“漫记”这部分,我们关注的焦点是在数学工作自身上。然而,我对这项工作发生的\textbf{背景}以及在真正的工作时间之外的生发的\textbf{动机}几乎只字未提。这样做的风险在于,有可能对我的个人形象(乃至于数学家和科学家们的一般形象)造成过誉但扭曲的效果,像是说什么“伟大而崇高的激情”——对此,不会有任何形式的纠正。简而言之,这将与所谓的“科学的神话”(\textbf{“科学”}要大写!)毫无区别。那些作家和学者一个比一个厉害地摔进(并且将会继续在同样的地方跌倒!)英雄主义的普罗米修斯神话的大坑。或许只有那些历史学家偶尔能够避开这个诱人的神话。事实是,在 "科学家 "的动机中,野心和虚荣心所起的作用与其他任何职业一样重要,而且几乎是普遍现象,这些动机有时促使他在工作中不计成本地投入。这在或精或粗的形式中得以展现,主要取决于相关的人员。当然,我也不例外。我希望读者在读我的证言\footnote{指作者对过去的见证。译者注}的时候不要对此产生疑惑。\par
我们说“最狂暴的野心也不能发现或是证明一个最简单的数学论断”,这其实也是对的,就好比(举例来说)“最狂暴的野心也不能让人硬得起来(就是字面意思)”一样。不管你是男人还是女人,真正能让你欲火焚身的绝不是那种野心,那种要炫耀自己(性)能力的野心——绝对不是这个!它应该是对于某些无比强烈、无比真实、无比微妙的事物的敏锐感知。我强名之曰“美”,这也是它的万千面目之一。确切地说,有野心并不妨碍我们感知“美”。但是,实话实话,我们能感知“美”\textbf{绝对不是}因为我们有野心。\par 
第一个驭火者绝对是和你我有相似之处的人类, 而绝对不是传说中的“大英雄”或者“半神”之类的存在。和你我一样,他一定体会过艰辛疲惫的煎熬,亦曾以狂妄自慰,这种安慰使他将先前历经的痛苦都抛之脑后。然而,在他终于燃起火种的那一刹那,所有的恐惧、所有的狂妄也就都不足为道了。这,就是所谓英雄神话的真实面貌。当这所谓的神话终究变成了我们探索事物另一面的阻碍之时(这另一面可能同样精要、同样真实),它就会变成索然无味的鸡肋、就会变成无聊的自我安慰之物。我在《收获与播种》中的目的是谈论这两个方面——关于求知的动力,以及关于“恐惧”和他的对立面“狂妄”。我认为我 “理解”,或者至少\textbf{了解},求知的动力和它的性质。(不过也没准,到哪天我会惊呼自己原来在自欺欺人。)但是,如果提到“恐惧”和“狂妄”,以及随之而来的、它们对创造力产生的险恶障碍,我自认为我还没有彻底弄清这个难题。甚至,我不知道自己能不能在有生之年彻悟这一大难题……\par
在撰写《收获与播种》的历程中,有两个意象浮现了出来。它们分别代表着人类冒险历程的两个方面。这两个意象就是\textbf{“孩童”}(别称:\textbf{工人})和\textbf{“老板”}。在接下来我们要着手的“漫记”部分,我们讨论的几乎全都是“孩童”。在\textbf{“孩童与母亲”}一节中,我们讨论的也是这个意象。 我们希望在“漫记”中散步的过程中,这个名称的真实含义能够更加清晰。\par 
在“反思”的其它部分,相应地,“老板”占据了论述的中心地位。他其实不是任何东西的老板!确切地说,我说的不是\textbf{一个}公司的老板,而是一系列竞争公司的老板们。然而,我们说所有公司的老板都彼此相似,这也是对的。而当我们说起“老板”的时候,可以肯定的是他们中一定有些“坏蛋”。在此介绍性部分(“四个乐章的前奏”)之后,紧接着就是我的“反思”的第一部分,“暗昧和新生”。这部分是主要关于我的,我就是那个“坏蛋”。而剩下的部分则主要是关于别人的。每个人轮流登场!\par 
这便是说,除了深刻的哲学反思以及(并非真正意义上的)“忏悔”,这里还会有一些“尖刻的肖像”(引用我同事的话,他发现他自己被这样粗暴对待了),更不用说有些不容易进行的大尺度“操作”了。Robert Jaulin \footnote{Robert Jaulin是我的老友。我理解的是,他在民族学界遭遇的状况(就像一匹“白狼”),和我在数学界“上流圈层”遭遇到的状况相似。} 半开玩笑地向我保证,我在《收获与播种》中做的是“数学界的民族学”(或者说社会学?我不知道怎么表达)。当然了,一个人如果(在不知情的情况下)从别人那知道自己居然在做一门学问,他会受宠若惊的!事实上,在反思的“调查”部分(令我懊恼的是......),我看到数学界的相当一部分人在我所写的页面上经过,更不用说一些地位较低的同事和朋友了。而最近几个月,自从去年10月我寄出《收获与播种》的临时印刷品后,这部分又得“回炉重造”了。我的证言就像掷入池水的大石,四面八方都有所回响(除了无聊的动静……)。然而,几乎每次扔石头,我听到的都不是我想要的。要么就是,我得到的反响只有寂静。这说明了很多问题。显然,我有(且仍有)很多东西需要学习。我要去了解我的那些前学生们和处境或好或坏的同事们脑子里缤纷多彩的想法,没错,我说的就是“数学圈里的社会学”。对于那些已经为我晚年的“社会学工作”做出贡献的人们,我在此向你们致以真诚的感谢。\par 
当然了,我对于那些温暖的回音是十分敏感的。当然,也有少有的几位同事向我表达了他们的一种灾难将至的忧心:他们感觉数学圈在日益败坏。\par 
在数学圈之外,从一开始就热烈欢迎我的“证言”的人,我希望在此处提及他们的名字:Sylvie et Catherine Chevalley \footnote{Sylvie Chevalley 和 Catherine Chevalley 是我的同事与好友Claude Chevalley的遗孀和女儿。《收获与播种》的核心(Res III, 阴阳之精要)就是献给他的。在我的“反思”中,我数次提起过他和他在我的行程中扮演的角色。} , Robert Jaulin, Stéphane Deligeorge, Christian Bourgois。如果《收获与播种》能够产生比先行本更大的影响力(先行本只是在我们一个小圈子里流动),那么这应该主要归功于他们。最重要的是,感谢他们的交流信念:我试图理解和说出来的东西必须要说出来。而且,这一点可以在我的小圈子之外更大的圈子里听到(小圈子里的人往往闷闷不乐,甚至很凶,根本不愿意质疑自己)。因此,Christian Bourgois毫不犹豫地承担了出版我这本根本不适合出版的作品所面临的风险。而Stéphane Deligeorge则给了我很大的荣幸:他把我晦涩难懂的“证言”收入了《书信集》之中,和(至少在当时)Newton、Cuvier、Arago并列。(我简直不能和比这更好的人们并列了!)感谢你们每一个人,感谢你们一再表达的同情和信任。在一个特别 "敏感 "的时刻,我很高兴能表达我的感激之情。\par 
现在,我们即将处于对我工作“漫记”的开端,这将是对我一生的旅程的一个介绍。确实,这是一段很漫长的旅程,足足有1000多页,并且每一页都封装得很好。我花了将近一辈子得时间走了这段路,直到现在也没有走完。我花了一年多的时间去重新体察这段路,一页接着一页。有时候文思姗姗来迟,我个人经历的甘露似乎在躲避语言的描述,就像有时候榨汁机里成熟而饱满的葡萄粒躲闪压榨的力量一样。但即便在那些语言似乎恣意奔流的时候,它们也没有就这样直接抵达至臻的境界。每句话,都要在当时或是事后被再度仔细权衡,如果它们太轻或者太重,我就会调整它们。这本“反思-证言-旅途”不是给那些急于求成的、希图一天或者一月就蹦到结语处的读者写的。《收获与播种》并\textbf{没有}什么“结语”或是“结论”之类的东西,它们会在你的、或我的生命中体现。有一种美酒,它在我生命的木桶中随着岁月而愈发醇厚,历久而弥香。你喝的最后一杯并不会比第一杯或者第一百杯好喝,它们实际上是“一样”的,却又不尽相同。如果第一杯就是坏的,那么整桶酒就都坏了;所以,与其喝坏酒,你还不如喝“好水”(如果真的有“好水”这种东西的话)。\par 
而真正的好酒,要慢慢品、细细品、用心品。
\end{document}